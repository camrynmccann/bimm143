% Options for packages loaded elsewhere
\PassOptionsToPackage{unicode}{hyperref}
\PassOptionsToPackage{hyphens}{url}
%
\documentclass[
]{article}
\title{Class11}
\author{Camryn McCann (PID: A15437387)}
\date{11/2/2021}

\usepackage{amsmath,amssymb}
\usepackage{lmodern}
\usepackage{iftex}
\ifPDFTeX
  \usepackage[T1]{fontenc}
  \usepackage[utf8]{inputenc}
  \usepackage{textcomp} % provide euro and other symbols
\else % if luatex or xetex
  \usepackage{unicode-math}
  \defaultfontfeatures{Scale=MatchLowercase}
  \defaultfontfeatures[\rmfamily]{Ligatures=TeX,Scale=1}
\fi
% Use upquote if available, for straight quotes in verbatim environments
\IfFileExists{upquote.sty}{\usepackage{upquote}}{}
\IfFileExists{microtype.sty}{% use microtype if available
  \usepackage[]{microtype}
  \UseMicrotypeSet[protrusion]{basicmath} % disable protrusion for tt fonts
}{}
\makeatletter
\@ifundefined{KOMAClassName}{% if non-KOMA class
  \IfFileExists{parskip.sty}{%
    \usepackage{parskip}
  }{% else
    \setlength{\parindent}{0pt}
    \setlength{\parskip}{6pt plus 2pt minus 1pt}}
}{% if KOMA class
  \KOMAoptions{parskip=half}}
\makeatother
\usepackage{xcolor}
\IfFileExists{xurl.sty}{\usepackage{xurl}}{} % add URL line breaks if available
\IfFileExists{bookmark.sty}{\usepackage{bookmark}}{\usepackage{hyperref}}
\hypersetup{
  pdftitle={Class11},
  pdfauthor={Camryn McCann (PID: A15437387)},
  hidelinks,
  pdfcreator={LaTeX via pandoc}}
\urlstyle{same} % disable monospaced font for URLs
\usepackage[margin=1in]{geometry}
\usepackage{color}
\usepackage{fancyvrb}
\newcommand{\VerbBar}{|}
\newcommand{\VERB}{\Verb[commandchars=\\\{\}]}
\DefineVerbatimEnvironment{Highlighting}{Verbatim}{commandchars=\\\{\}}
% Add ',fontsize=\small' for more characters per line
\usepackage{framed}
\definecolor{shadecolor}{RGB}{248,248,248}
\newenvironment{Shaded}{\begin{snugshade}}{\end{snugshade}}
\newcommand{\AlertTok}[1]{\textcolor[rgb]{0.94,0.16,0.16}{#1}}
\newcommand{\AnnotationTok}[1]{\textcolor[rgb]{0.56,0.35,0.01}{\textbf{\textit{#1}}}}
\newcommand{\AttributeTok}[1]{\textcolor[rgb]{0.77,0.63,0.00}{#1}}
\newcommand{\BaseNTok}[1]{\textcolor[rgb]{0.00,0.00,0.81}{#1}}
\newcommand{\BuiltInTok}[1]{#1}
\newcommand{\CharTok}[1]{\textcolor[rgb]{0.31,0.60,0.02}{#1}}
\newcommand{\CommentTok}[1]{\textcolor[rgb]{0.56,0.35,0.01}{\textit{#1}}}
\newcommand{\CommentVarTok}[1]{\textcolor[rgb]{0.56,0.35,0.01}{\textbf{\textit{#1}}}}
\newcommand{\ConstantTok}[1]{\textcolor[rgb]{0.00,0.00,0.00}{#1}}
\newcommand{\ControlFlowTok}[1]{\textcolor[rgb]{0.13,0.29,0.53}{\textbf{#1}}}
\newcommand{\DataTypeTok}[1]{\textcolor[rgb]{0.13,0.29,0.53}{#1}}
\newcommand{\DecValTok}[1]{\textcolor[rgb]{0.00,0.00,0.81}{#1}}
\newcommand{\DocumentationTok}[1]{\textcolor[rgb]{0.56,0.35,0.01}{\textbf{\textit{#1}}}}
\newcommand{\ErrorTok}[1]{\textcolor[rgb]{0.64,0.00,0.00}{\textbf{#1}}}
\newcommand{\ExtensionTok}[1]{#1}
\newcommand{\FloatTok}[1]{\textcolor[rgb]{0.00,0.00,0.81}{#1}}
\newcommand{\FunctionTok}[1]{\textcolor[rgb]{0.00,0.00,0.00}{#1}}
\newcommand{\ImportTok}[1]{#1}
\newcommand{\InformationTok}[1]{\textcolor[rgb]{0.56,0.35,0.01}{\textbf{\textit{#1}}}}
\newcommand{\KeywordTok}[1]{\textcolor[rgb]{0.13,0.29,0.53}{\textbf{#1}}}
\newcommand{\NormalTok}[1]{#1}
\newcommand{\OperatorTok}[1]{\textcolor[rgb]{0.81,0.36,0.00}{\textbf{#1}}}
\newcommand{\OtherTok}[1]{\textcolor[rgb]{0.56,0.35,0.01}{#1}}
\newcommand{\PreprocessorTok}[1]{\textcolor[rgb]{0.56,0.35,0.01}{\textit{#1}}}
\newcommand{\RegionMarkerTok}[1]{#1}
\newcommand{\SpecialCharTok}[1]{\textcolor[rgb]{0.00,0.00,0.00}{#1}}
\newcommand{\SpecialStringTok}[1]{\textcolor[rgb]{0.31,0.60,0.02}{#1}}
\newcommand{\StringTok}[1]{\textcolor[rgb]{0.31,0.60,0.02}{#1}}
\newcommand{\VariableTok}[1]{\textcolor[rgb]{0.00,0.00,0.00}{#1}}
\newcommand{\VerbatimStringTok}[1]{\textcolor[rgb]{0.31,0.60,0.02}{#1}}
\newcommand{\WarningTok}[1]{\textcolor[rgb]{0.56,0.35,0.01}{\textbf{\textit{#1}}}}
\usepackage{graphicx}
\makeatletter
\def\maxwidth{\ifdim\Gin@nat@width>\linewidth\linewidth\else\Gin@nat@width\fi}
\def\maxheight{\ifdim\Gin@nat@height>\textheight\textheight\else\Gin@nat@height\fi}
\makeatother
% Scale images if necessary, so that they will not overflow the page
% margins by default, and it is still possible to overwrite the defaults
% using explicit options in \includegraphics[width, height, ...]{}
\setkeys{Gin}{width=\maxwidth,height=\maxheight,keepaspectratio}
% Set default figure placement to htbp
\makeatletter
\def\fps@figure{htbp}
\makeatother
\setlength{\emergencystretch}{3em} % prevent overfull lines
\providecommand{\tightlist}{%
  \setlength{\itemsep}{0pt}\setlength{\parskip}{0pt}}
\setcounter{secnumdepth}{-\maxdimen} % remove section numbering
\ifLuaTeX
  \usepackage{selnolig}  % disable illegal ligatures
\fi

\begin{document}
\maketitle

\#Introduction to the RCSB Protein Data Bank (PDB)

Downloaded the following CSV dile from the PDB site.

\begin{Shaded}
\begin{Highlighting}[]
\NormalTok{db }\OtherTok{\textless{}{-}} \FunctionTok{read.csv}\NormalTok{(}\StringTok{"Data Export Summary.csv"}\NormalTok{, }\AttributeTok{row.names =} \DecValTok{1}\NormalTok{)}
\FunctionTok{head}\NormalTok{(db)}
\end{Highlighting}
\end{Shaded}

\begin{verbatim}
##                          X.ray   NMR   EM Multiple.methods Neutron Other  Total
## Protein (only)          142303 11804 5999              177      70    32 160385
## Protein/Oligosaccharide   8414    31  979                5       0     0   9429
## Protein/NA                7491   274 1986                3       0     0   9754
## Nucleic acid (only)       2368  1372   60                8       2     1   3811
## Other                      149    31    3                0       0     0    183
## Oligosaccharide (only)      11     6    0                1       0     4     22
\end{verbatim}

\begin{quote}
\textbf{Q1: What percentage of structures in the PDB are solved by X-Ray
and Electron Microscopy.}
\end{quote}

\begin{Shaded}
\begin{Highlighting}[]
\FunctionTok{round}\NormalTok{(}\FunctionTok{sum}\NormalTok{(db}\SpecialCharTok{$}\NormalTok{X.ray)}\SpecialCharTok{/}\FunctionTok{sum}\NormalTok{(db}\SpecialCharTok{$}\NormalTok{Total)}\SpecialCharTok{*}\DecValTok{100}\NormalTok{,}\DecValTok{2}\NormalTok{)}
\end{Highlighting}
\end{Shaded}

\begin{verbatim}
## [1] 87.55
\end{verbatim}

\begin{Shaded}
\begin{Highlighting}[]
\FunctionTok{round}\NormalTok{(}\FunctionTok{sum}\NormalTok{(db}\SpecialCharTok{$}\NormalTok{EM)}\SpecialCharTok{/}\FunctionTok{sum}\NormalTok{(db}\SpecialCharTok{$}\NormalTok{Total)}\SpecialCharTok{*}\DecValTok{100}\NormalTok{,}\DecValTok{2}\NormalTok{)}
\end{Highlighting}
\end{Shaded}

\begin{verbatim}
## [1] 4.92
\end{verbatim}

\begin{quote}
\textbf{Q2: What proportion of structures in the PDB are protein?}
\end{quote}

\begin{Shaded}
\begin{Highlighting}[]
\FunctionTok{round}\NormalTok{(db}\SpecialCharTok{$}\NormalTok{Total[}\DecValTok{1}\NormalTok{]}\SpecialCharTok{/}\FunctionTok{sum}\NormalTok{(db}\SpecialCharTok{$}\NormalTok{Total)}\SpecialCharTok{*}\DecValTok{100}\NormalTok{,}\DecValTok{2}\NormalTok{)}
\end{Highlighting}
\end{Shaded}

\begin{verbatim}
## [1] 87.36
\end{verbatim}

\begin{quote}
\textbf{Q3: Type HIV in the PDB website search box on the home page and
determine how many HIV-1 protease structures are in the current PDB?}
\end{quote}

????

\#Visualizing the HIV-1 protease structure

\begin{quote}
\textbf{Q4: Water molecules normally have 3 atoms. Why do we see just
one atom per water molecule in this structure?}
\end{quote}

These water molecules only have one atom, which is oxygen, because the 2
hydrogen atoms present are too small to see.

\begin{quote}
\textbf{Q5: There is a conserved water molecule in the binding site. Can
you identify this water molecule? What residue number does this water
molecule have (see note below)?}
\end{quote}

\includegraphics{vmdscene.tga}

The water molecule is conserved at residue 308.

\#Introduction to Bio3D in R

\begin{Shaded}
\begin{Highlighting}[]
\FunctionTok{library}\NormalTok{(bio3d)}
\end{Highlighting}
\end{Shaded}

\begin{Shaded}
\begin{Highlighting}[]
\NormalTok{pdb }\OtherTok{\textless{}{-}} \FunctionTok{read.pdb}\NormalTok{(}\StringTok{"1hsg.pdb"}\NormalTok{)}
\NormalTok{pdb}
\end{Highlighting}
\end{Shaded}

\begin{verbatim}
## 
##  Call:  read.pdb(file = "1hsg.pdb")
## 
##    Total Models#: 1
##      Total Atoms#: 1686,  XYZs#: 5058  Chains#: 2  (values: A B)
## 
##      Protein Atoms#: 1514  (residues/Calpha atoms#: 198)
##      Nucleic acid Atoms#: 0  (residues/phosphate atoms#: 0)
## 
##      Non-protein/nucleic Atoms#: 172  (residues: 128)
##      Non-protein/nucleic resid values: [ HOH (127), MK1 (1) ]
## 
##    Protein sequence:
##       PQITLWQRPLVTIKIGGQLKEALLDTGADDTVLEEMSLPGRWKPKMIGGIGGFIKVRQYD
##       QILIEICGHKAIGTVLVGPTPVNIIGRNLLTQIGCTLNFPQITLWQRPLVTIKIGGQLKE
##       ALLDTGADDTVLEEMSLPGRWKPKMIGGIGGFIKVRQYDQILIEICGHKAIGTVLVGPTP
##       VNIIGRNLLTQIGCTLNF
## 
## + attr: atom, xyz, seqres, helix, sheet,
##         calpha, remark, call
\end{verbatim}

\#Comparative structure analysis of Adenylate Kinase

\end{document}
